

\chapter{Expression of Angular Eigenstates of Harmonic Oscillators} \label{appA}

In this appendix, we derive some results on harmonic oscillators which are useful to show the abilities of entanglement detection of the SRPT criterion.

\section{Two Dimensional Harmonic Oscillator}

In two dimensions, the Hamiltonian describing a particle of mass $m$ subject to an isotropic harmonic potential is

\[ H =  \frac{p_x^2}{2m}+\frac{p_y^2}{2m} + \frac 1 2 m \omega^2 \left( x^2 + y^2 \right),\]
with $x_i, p_i$ the position and momentum of the particle in the directions $i=x,y$ and $\omega$ a constant. Since $H$ is the sum of two  oscillators Hamiltonians along the directions $x$ and $y$, it is clear that the eigenvalues of the hamiltonian will be written as
\[ \ket{n_x, n_y} \equiv \ket{n_x} \otimes \ket{n_y}, \]
when the $\ket{n_i}$ are the eigenstates of the two one dimensional Hamiltonians. We may define the annihilation operators:
\begin{eqnarray}
a_x &=& \frac{1}{\sqrt 2} \left( \sqrt{\frac{m\omega}{\hbar}} x + \frac{i}{\sqrt{m \hbar \omega}} p_x \right), \\ 
a_y &=& \frac{1}{\sqrt 2} \left( \sqrt{\frac{m\omega}{\hbar}} y +\frac{i}{\sqrt{m \hbar \omega}} p_y \right).
\end{eqnarray}
The creation operators $a^\dagger_i$ are the hermitian conjugates of the above operators. We have
\[ [a_x,a^\dagger_x]= [a_y,a^\dagger_y]=1, \]
while all other commutators between the four operators are zero. Let us finally define the number operators
\begin{eqnarray}
N_x &=&a_x^\dagger a_x, \\ 
N_y &=& a_y^\dagger a_y,
\end{eqnarray}
which allow us to write $H$ as
\[ H = (N_x + N_y +1) \hbar \omega. \]

The eigenstates have the properties
\begin{eqnarray}
a_i \ket{n_i} &=& \sqrt{n_i} \ket{n_i-1},\\
a_i^\dagger \ket{n_i} &=& \sqrt{n_i+1} \ket{n_i+1},\\
N_i \ket{n_i} &=& n_i \ket{n_i},
\end{eqnarray}
and therefore, we have
\[  \ket{n_x,n_y} = \frac{1}{\sqrt{n_x! n_y!}} \left( a_x^\dagger \right)^{n_x}  \left( a_y^\dagger \right)^{n_y} \ket{0,0}, \]
with $\ket{0,0}$ the ground state of the oscillator.

We can also prove that
\[ H \ket{n_x,n_y} = (n_x + n_y +1) \hbar \omega \ket{n_x,n_y}.\]

If we define the total quantum number
\[ n = n_x + n_y, \]
we note that one particular value of $n$ corresponds to the $n+1$ orthogonal eigenstates
\[ \ket{n,0}, \; \ket{n-1,1}, \ldots, \; \ket{0,n}, \]
which shows that the measure of the energy alone does not allow us to pinpoint one proper state. The discrimination between all these states could be achieved by measuring the energy separately in the $x$ and $y$ direction, but there is another method that takes advantage of the angular momentum $L_z$. That operator is defined by
\[ L_z = x p_y - y p_x,\]
or in other terms
\[ L_z= i\hbar(a_x a_y^\dagger-a_x^\dagger a_y), \]
and it is clear that the $\ket{n_x,n_y}$ states are not eigenstates of $L_z$.

However, it is possible to show that 
\[ [H,L_z]=0, \]
which indicates there must be a common eigenstate basis for the two operators. That basis is made of the $\ket{\psi_{k,M}}$ eigenvectors which has the following properties:
\begin{eqnarray}
H \ket{\psi_{k,M}} &=& \hbar \omega(2k+|M|+1) \ket{\psi_{k,M}},\\
L_z \ket{\psi_{k,M}} &=& \hbar M \ket{\psi_{k,M}}.
\end{eqnarray}

We wish to express any $\ket{\psi_{k,M}}$ state in function of the $\ket{n_x,n_y}$ states. It is quite clear that $n=n_x+n_y=2k+|M|$ and therefore for given $k$ and $M$ we have in general
\[ \ket{\psi_{k,M}} = \sum_{i=0}^n c_i \ket{n_x=n-i, n_y=i}. \label{eq-2Dsum}\]

To find the analytical expression of the $c_i$ coefficients, we need to define an intermediate basis as follow
\begin{eqnarray}
a_r &=& \frac{1}{\sqrt 2} \left(a_x-ia_y \right), \\ 
a_l &=& \frac{1}{\sqrt 2} \left(a_x+ia_y \right),
\end{eqnarray}
where $r$ and $l$ stand for right and left, as those operators can be interpreted as annihilators of right and left  ``circular quanta''. They are very similar to $a_x$ and $a_y$, they follow the commutation rules
\[ [a_r,a^\dagger_r]= [a_l,a^\dagger_l]=1, \]
and all other combinations are zero. It is possible to express the regular ladder operators in function of the circular one and we find that
\begin{eqnarray}
H&=&(N_r + N_l +1) \hbar \omega,\\
L_z &=& \hbar (N_r- N_l) ,
\end{eqnarray}
with $N_r=a^\dagger_r a_r$ and $N_l=a^\dagger_l a_l$ the new number operators. We see that there is a $\ket{\varphi_{n_r,n_l}}$ basis in which both $H$ and $L_z$ are diagonal and which behaves exactly like the $\ket{n_x,n_y}$ basis. We also find
\begin{eqnarray}
n &=& n_r + n_l = 2k+ |M|, \\ 
M &=& n_r - n_l,
\end{eqnarray}

This result does allow to associate a definite $\ket{\psi_{k,M}}$ state to a single state $\ket{\varphi_{n_r,n_l}}$. There are two cases depending on the sign of $M$; if $M>0$, it means $n_r>n_l$ and $|M|=n_r-n_l$ and $k=n_l$, if $M<0$, then $|M|=n_l-n_r$ and $k=n_r$. Globally,
\begin{eqnarray}
\ket{\psi_{k,M}} &=& \ket{\varphi_{n_r=k+|M|,n_l=k}} \:\: \textrm{if } M>0, \\ 
\ket{\psi_{k,M}} &=& \ket{\varphi_{n_r=k,n_l=k+|M|}} \:\: \textrm{if } M<0.
\end{eqnarray}

We see that a state with a positive helicity $M$ has more ``right'' quanta than ``left'' quanta and inversely for $M<0$. Now, to express a $\ket{\varphi_{n_r,n_l}}$ into a combination of $\ket{n_x,n_y}$ we have
\begin{eqnarray}
\ket{\varphi_{n_r,n_l}}&=& \frac{1}{\sqrt{n_r! n_l!}} \left( a_r^\dagger \right)^{n_r}  \left( a_l^\dagger \right)^{n_l} \ket{\varphi_{0,0}}, \\
&=&   \frac{1}{\sqrt{2^n n_r! n_l!}} \left( a_x^\dagger +ia_y^\dagger \right)^{n_r}  \left( a_x^\dagger -ia_y^\dagger\right)^{n_l} \ket{0,0}, \label{eq-2D2}\\
&=&   \frac{1}{\sqrt{2^n n_r! n_l!}} \sum_{k=0}^{n_r} \sum_{l=0}^{n_l}  \binom{n_r}{k} \binom{n_l}{l} \left( a_x^\dagger \right)^{n-k-l}  \left(a_y^\dagger \right)^{j+k} (i)^{k-l} \ket{0,0}, \label{eq-2D3}\\
&=&   \frac{1}{\sqrt{2^n n_r! n_l!}} \sum_{i=0}^{n} \sum_{j}^{\{-i,-i+2,\cdots, i\}}  \binom{n_r}{\frac{i-j}{2}} \binom{n_l}{\frac{i+j}{2}} \left( a_x^\dagger \right)^{n-i}  \left(a_y^\dagger \right)^i (i)^{j} \ket{0,0}, \label{eq-2D4}\\
&=&   \frac{1}{\sqrt{2^n n_r! n_l!}} \sum_{i=0}^{n} \sum_{j=0}^{i}  \binom{n_r}{j} \binom{n_l}{i-j} \left( a_x^\dagger \right)^{n-i}  \left(a_y^\dagger \right)^i (i)^{2j-i} \ket{0,0},\label{eq-2D5} \\
&=&   \sum_{i=0}^{n}(-i)^i  \sqrt{ \frac{(n-i)! i!}{2^n n_r! n_l!}}    \sum_{j=0}^{i} (-1)^j \binom{n_r}{j} \binom{n_l}{i-j}  \ket{n-i,i},\label{eq-2D6} \\
&=&   \sum_{i=0}^{n}(-i)^i  \sqrt{\frac{\binom{n}{n_r} }{2^n \binom{n}{i}}}    \sum_{j=0}^{i}(-1)^j \binom{n_r}{j} \binom{n_l}{i-j}  \ket{n-i,i}, \label{eq-2D7}
\end{eqnarray}
where on line (\ref{eq-2D2}) we simply used the definition of $a_r$ and $a_l$ and noted the ground state is the same in both basis, in (\ref{eq-2D3}) we used the binomial formula twice and in (\ref{eq-2D4}) we applied the change of variables $i=k+l$ and $j=k-l$. In order to span all values of $(k,l)$ only once, we span all ``antidiagonal'' lines with $k+l=i$ $(i=0,1,\ldots,n)$ and along those lines we consider the elements $(k=\frac{i+j}{2},l=\frac{i-j}{2})$ $(j=-i, -i+2,\ldots, i-2,i)$. The values of $j$ should actually only go from $\max\{-i,i-2 n_l\}$ to  $\min\{i,2 n_r -i\}$ in order not to go beyond $k=n_r$ and $l=n_l$ but we can simplify it since the binomial coefficients will yield zero if $k>n_r$ or $l>n_l$. In line (\ref{eq-2D5}) we applied yet another change of variable as $j'=\frac{i+j}{2}$ $(j=0, 1, \ldots, i)$ and renamed $j'$ as $j$, in (\ref{eq-2D6}) we applied the creation operators to the ground state and in (\ref{eq-2D7}) we multiplied the numerator and denominator in the root term by $n!$ and simplified the expression. The final sum on $j$ may be expressed differently as we find that
\[  \sum_{j=0}^{i}(-1)^j \binom{n_r}{j} \binom{n_l}{i-j} = \binom{n_l}{i} {}_2F_1(-i,-n_r; n_l -i+1;-1), \]
with ${}_2F_1$ the hypergeometric function, but we choose to keep the sum as it is for its implementation simplicity.

Now,  we need to express directly the $\ket{\psi_{k,n}}$ state in the $\ket{n_x, n_y}$ basis. Depending on the sign of $M$, we need to consider the state $\ket{\varphi_{k+|M|, k}}$ or $\ket{\varphi_{k,k+|M|}}$. The easiest way to see the effect of a swap of $n_r$ and $n_l$ is in line (\ref{eq-2D3}) where the only difference is $i^{k-l}$  becoming $i^{l-k}=(-i)^{k-l}$. With that consideration, we finally have the $c_i$ coefficients we wanted in (\ref{eq-2Dsum}) up to an overall phase:
\[ c_i =  \left(- \textrm{sign}(M) i \right)^i  \sqrt{\frac{\binom{n}{k} }{2^n \binom{n}{i}}}    \sum_{j=0}^{i}(-1)^j \binom{k+|M|}{j} \binom{k}{i-j}.\]

The last property we want to investigate is the relation
\[ c_i = i^n (-1)^{n_l-i} c_{n-i} . \]
On the left hand side, we have
\begin{eqnarray}
c_i&=& (-i)^i  \sqrt{\frac{\binom{n}{n_r} }{2^n \binom{n}{i}}}  \sum_{j=0}^{i}(-1)^j \binom{n_r}{j} \binom{n_l}{i-j} ,
\end{eqnarray}
and on the right hand side
\begin{eqnarray}
c_{n-i}&=&  (-i)^{n-i} \sqrt{\frac{\binom{n}{n_r} }{2^n \binom{n}{n-i}}}    \sum_{j=0}^{n-i}(-1)^j \binom{n_r}{j} \binom{n_l}{n-i-j} , \\
&=&  (-i)^{n-i}  \sqrt{\frac{\binom{n}{n_r} }{2^n \binom{n}{i}}}    \sum_{j=n_l-i}^{n_r}(-1)^{n_r-j} \binom{n_r}{n_r-j} \binom{n_l}{n-i+j-n_r} , \\
&=&  (-i)^{n-i} (-1)^{n_r}  \sqrt{\frac{\binom{n}{n_r} }{2^n \binom{n}{i}}}    \sum_{j=n_l-i}^{n_r}(-1)^{j} \binom{n_r}{j} \binom{n_l}{i-j} ,
\end{eqnarray}
where in the second step we used the variable change $j'=n_r-j$ and in the last step simplified the expression. Aside from the phase, the only remaining difference is the borns of the sum but thanks to the binomial coefficients, that difference vanishes. Indeed, for both sums, the condition of having non zero terms is $\max\{ 0, i- n_l\} \le j \le \min\{ i,n_r \} $, so that all terms considered outside those limits are zero. Of course, we have $|c_i|=|c_{n-i}|$.


\section{Three Dimensional Harmonic Oscillator}

In three dimensions, the Hamiltonian describing a particle of mass $m$ subject to an isotropic harmonic potential is

\[ H =  \frac{p_x^2}{2m}+\frac{p_y^2}{2m} + \frac{p_z^2}{2m}+ \frac 1 2 m \omega^2 \left( x^2 + y^2 +z^2\right),\]
with $x_i, p_i$ the position and momentum of the particle in the directions $i=x,y,z$ and $\omega$ a constant. Once again, since $H$ is the sum of three oscillators Hamiltonians, the eigenvalues of the hamiltonian will be written as
\[ \ket{n_x, n_y,n_z} \equiv \ket{n_x} \otimes \ket{n_y} \otimes \ket{n_z}, \]
when the $\ket{n_i}$ are the eigenstates of the three one dimensional Hamiltonians. We may define the third annihilation operator:
\[ a_z = \frac{1}{\sqrt 2} \left( \sqrt{\frac{m\omega}{\hbar}} z + \frac{i}{\sqrt{m \hbar \omega}} p_z \right),\]
which behaves exactly as the others. We also define its number operator
\[ N_z = a_z^\dagger a_z, \]
which allow us to write $H$ as
\[ H = \left(N_x + N_y + N_z +\frac 3 2 \right) \hbar \omega. \]

The eigenstates are of the form
\[  \ket{n_x,n_y,n_z} = \frac{1}{\sqrt{n_x! n_y!n_z!}} \left( a_x^\dagger \right)^{n_x}  \left( a_y^\dagger \right)^{n_y} \left( a_z^\dagger \right)^{n_z} \ket{0,0,0}, \]
with $\ket{0,0,0}$ the ground state of the oscillator. This time for a definite energy $n=n_x+n_y+n_z$ there is a degree of degenerescence $g_n$ of
\[ g_n = \sum_{i=0}^n (i+1) = \frac{n(n+1)}{2} + (n+1) = \frac 1 2 (n+1)(n+2). \]
Just as we introduced $L_z$ in the two dimensional harmonic oscillator, we introduce the additional observable
\[ \mathbf{L}^2 = \frac 1 2 (L_+ L_- + L_- L_+) + L_z^2, \]
with
\begin{eqnarray}
L_+ &=& \hbar \sqrt 2 (a^\dagger_z a_l - a^\dagger_r a_z), \\ 
L_- &=& \hbar \sqrt 2 (a^\dagger_l a_z - a^\dagger_z a_r). 
\end{eqnarray}

It can be checked that $[H,\mathbf{L}^2]=0$ and $[L_z,\mathbf{L}^2]=0$ which implies there must be a common set of eigenstates for $H$, $L_z$ and $\mathbf{L}^2$ . Those eigenstates are the $\ket{\psi_{k,l,m}}$ states and have the following properties
\begin{eqnarray}
H \ket{\psi_{k,l,m}} &=& \hbar \omega(2k+l+\frac 3 2) \ket{\psi_{k,l,m}},\\
L_z \ket{\psi_{k,l,m}} &=& \hbar m \ket{\psi_{k,l,m}},\\
\mathbf{L}^2 \ket{\psi_{k,l,m}} &=& \hbar^2 l(l+1) \ket{\psi_{k,l,m}}, \\
L_\pm \ket{\psi_{k,l,m}} &=& \hbar \sqrt{l(l+1)-m(m\pm1)} \ket{\psi_{k,l,m\pm1}}.\\
\end{eqnarray}

The last relation implies that $m$ can take any integer value from $-l$ to $l$. So by repeatedly applying the $L_-$ operator on a $\ket{\psi_{k,l,l}}$ state, we should be able to generate all states down to $\ket{\psi_{k,l,-l}}$. The first step is to find the expression of the $\ket{\psi_{k,l,l}}$ in the $\ket{\varphi_{n_r,n_l,n_z}}$ basis using the particular property $L_+ \ket{\psi_{k,l,l}} = 0$. By identifying the quantum numbers we find
\begin{eqnarray}
n&=& n_r + n_l + n_z = 2k+ l, \\ 
m &=& n_r - n_l.
\end{eqnarray}

Since $n_r-n_l=m$ we can always write $n_r = K+m$, $n_r=K$ with $K$ a positive integer and therefore we must have $n_z=n-m-2K$, which cannot be negative hence we have $K \le \frac{n-m}{2}$.  We are now able to write
\[ \ket{\psi_{k,l,l}} = \sum_{K=0}^k c_K \ket{\varphi_{n_r=K+l, n_l=K, n_z = 2k-2K}}, \]
since $\frac{n-l}{2} = k$ and with normalized coefficient $c_K$. The effect of $L_+$ on such a decomposition is
\bea
L_+ \ket{\psi_{k,l,l}} &=& \sum_{K=0}^k c_K \sqrt 2 \hbar \left( \sqrt{K} \sqrt{2k-2K+1} \ket{\varphi_{K+l, K-1,2k-2K+1}} \right. \nonumber \\
&& \ - \left. \sqrt{K+l+1}\sqrt{2k-2K} \ket{\varphi_{K+l+1, K,2k-2K-1}} \right), 
\eea
which is zero if 
\[ c_K  \sqrt{K+l+1}\sqrt{2k-2K} = c_{K+1} \sqrt{K+1} \sqrt{2k-2K-1}. \]

From that relation, we can express all $c_K$ in function of $c_0$ and then normalize all coefficient. We have
\bea
c_K^2 &=& c_0^2 \prod_{j=0}^{K-1} \frac{(j+l+1)(2k-2j)}{(j+1)(2k-2j-1)} ,\\
&=& c_0^2 \frac{(l+1)(l+2)\ldots(l+K)(2k)(2k-2)\ldots(2k-2K+2)}{(1)(2)\ldots K (2k-1)(2k-3)\ldots(2k-2K+1)} ,\\
&=& c_0^2 \frac{(l+K)!}{l!K! } \frac{(2k)!!}{(2k-2K)!!} \frac{(2k-2K-1)!!}{(2k-1)!!}, \\
&=& c_0^2 \binom{l+K}{K} \frac{2^k k! }{2^{k-K}(k-K)! }\frac{(2k-2K)!}{2^{k-K}(k-K)! }\frac{2^{k}(k)! }{(2k)!}, \\
&=& c_0^2 \,2^{2K}\,\frac{ \binom{l+K}{K} \binom{2k-2K}{k-K}  }{ \binom{2k}{k} }, 
\eea
where $k!!$ is the double factorial of $k$ which has the property
\bea
(2k)!! &=&(2k)(2k-2)\ldots2 = k! 2^{k}, \\
(2k+1)!! &=&(2k+1)(2k-1)\ldots1 = \frac{(2k)!}{k! 2^{k}},
\eea
that we used in the process. 

With that expression, we are now able to calculate the value of $c_0$ by normalizing the expression. We have
\bea
\sum_{K=0}^k c_K^2 &=& c_0^2 \sum_{K=0}^k \,2^{2K}\,\frac{ \binom{l+K}{K} \binom{2k-2K}{k-K}  }{ \binom{2k}{k} }, \\
&=& c_0^2 \frac{ \binom{2(k+l+1)}{2k}}{ \binom{k+l+1}{k} },
\eea
Since that expression must be 1, we have he value of $c_0$. Finally, we get up to a global phase
\[ \ket{\psi_{k,l,l}} = \sum_{K=0}^k\,2^{K}\, \sqrt{ \frac{ \binom{l+K}{K} \binom{2k-2K}{k-K} \binom{k+l+1}{k} }{ \binom{2k}{k}  \binom{2(k+l+1)}{2k} }} \ket{\varphi_{K+l, K, 2k-2K}}. \]

%From that state we can get any other  $\ket{\psi_{k,l,m}}$ by applying $L_-$ the right number of time. Indeed, we have
%\[ \ket{\psi_{k,l,m}}=\frac{(L_-/\hbar)^{l-m}}{\prod_{j=m+1}^l \sqrt{l(l+1)-j(j-1)}} \ket{\psi_{k,l,l}}. \]
%
%First, we simplify the product at the denominator
%\bea
%\prod_{j=m+1}^l \sqrt{l(l+1)-j(j-1)} &=& \prod_{j=m+1}^l \sqrt{l+j}\sqrt{l-j+1}, \\
%&=& \sqrt{\frac{(2l)!}{(l+m)!}}\sqrt{(l-m)!}=\sqrt{\binom{2l}{l+m}} (l-m)!.
%\eea
%We also have
%\[ \left(\frac{L_-}{\hbar}\right)^{l-m} = 2^{\frac{l-m}{2}} (a^\dagger_l a_z - a^\dagger_z a_r)^{l-m} = 2^{\frac{l-m}{2}} \sum_{j=0}^{l-m} (-1)^j \binom{l-m}{j} (a^\dagger_l a_z)^{l-m-j} (a^\dagger_z a_r)^j, \]
%and by noting that
%\bea 
%a^{k} \ket{n} &=& \sqrt\frac{n!}{(n-k)!} \ket{n-k}, \\ 
%(a^\dagger)^{k} \ket{n} &=& \sqrt\frac{(n+k)!}{n!} \ket{n+k}, 
%\eea
%we see that
%\bea 
%(a^\dagger_l a_z)^{l-m-j} (a^\dagger_z a_r)^j  \ket{\varphi_{K+l, K, 2k-2K}} &=&   \sqrt\frac{(K+l)!}{(K+l-j)!} \sqrt\frac{(K+l-m-j)!}{K!} \nonumber \\
%&& \ \sqrt\frac{(2k-2K+j)!}{(2k-2K-l+m+2j)!} \sqrt\frac{(2k-2K+j)!}{(2k-2K)!} \nonumber \\
%&& \ \ket{\varphi_{K+l-j, K+l-m-j, 2k-2K-l+m+2j}},
%\eea
%and if we apply the variable change $j'=K+l-j$ ($j'=K+m,\ldots, K+l$)
%\[ (a^\dagger_l a_z)^{l-m-j} (a^\dagger_z a_r)^j  \ket{\varphi_{K+l, K, 2k-2K}} =   \sqrt\frac{(K+l)!(j-m)!(n-K-j)!^2}{j!K!(n+m-2j)!(2k-2K)!} \ket{\varphi_{j, j-m, n+m-2j}}.\]
%
%The expression became
%\[ \ket{\psi_{k,l,m}}=\sum_{K=0}^k   \sum_{j=K+m}^{K+l}  (-1)^j  c_{K,j}\ket{\varphi_{j, j-m, n+m-2j}},\]
%with
%\bea
%c_{K,j}&=&  \sqrt{ \frac{  \binom{l-m}{K+l-j}^2  \binom{l+K}{K} \binom{2k-2K}{k-K} \binom{k+l+1}{k} }{2^{l-m-2K} \binom{2k}{k}  \binom{2(k+l+1)}{2k} \binom{2l}{l+m}}}  \sqrt\frac{(K+l)!(j-m)!(n-K-j)!^2}{(l-m)!^2j!K!(n+m-2j)!(2k-2K)!} 
%\eea
